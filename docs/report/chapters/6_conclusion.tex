\chapter{Conclusiones y Trabajos Futuros}

La estimación de la edad de la muerte constituye una parte esencial del PB y, al mismo tiempo, un problema complejo de gran relevancia para la AF, debido a la subjetividad inherente a los métodos manuales actualmente empleados por los expertos. Este TFM abordó la clasificación automática de los criterios morfológicos utilizados para dicha estimación por medio de la sínfisis del pubis, con el objetivo de obtener modelos capaces de automatizar la extracción de estos criterios y asistir al experto humano en la toma de decisiones.

En primera instancia, se llevó a cabo un estudio detallado de la literatura sobre estimación de edad a partir de restos óseos, con foco en la sínfisis del pubis, así como del procesamiento de modelos 3D mediante DL. Se observó que la AF continúa utilizando mayoritariamente el método de Todd \cite{RefWorks:RefID:19-todd1921age} o variantes del mismo, con escasa incorporación de innovaciones tecnológicas basadas en DL. En cuanto al tratamiento de datos tridimensionales, se constató la falta de consenso sobre su representación más adecuada, existiendo múltiples alternativas. Para este trabajo se optó por el uso de mallas poligonales, por tratarse de un formato ampliamente utilizado tanto en informática gráfica como en el trabajo cotidiano de los antropólogos forenses. A su vez, no se hallaron estudios previos que aborden la clasificación directa y automática de las características morfológicas del pubis como los aquí planteados.

Se exploraron diversas propuestas metodológicas basadas en mallas poligonales, una representación relativamente novedosa en este campo, y se seleccionó el enfoque ExMeshCNN como \textit{framework} base para el diseño de los modelos, aplicando NAS mediante la librería Optuna \cite{optuna_2019} para optimizar su estructura y demás hiperparámetros relevantes.

El desarrollo de modelos implicó la creación de herramientas específicas para el preprocesamiento de los datos, principalmente la reducción de complejidad topológica mediante colapso de aristas, procurando minimizar la pérdida de información, así como el sellado automático de las mallas. Como parte de los experimentos, se llevaron a cabo 83 ejecuciones que incluyeron un total de 10,800 entrenamientos de modelos entrenados en una etiqueta y 6,000 entrenamientos con múltiples etiquetas por modelo, sumando 16,800 entrenamientos en total. Se obtuvieron así los 9 mejores modelos para cada una de las características del método de Todd, todos con puntuaciones de F1 Macro superiores a 0.6, alcanzando un F1 Macro promedio de 0.6891 y superando el 0.7 en 4 de las 9 características, con una característica obteniendo un valor de 0.92. Cabe destacar que esta métrica (F1 Macro) es conservadora, y el análisis de las matrices de confusión evidencia que los modelos son capaces de clasificar correctamente la mayoría de las muestras, incluso considerando la limitada cantidad de datos y el desbalance entre clases. De igual modo, se obtuvieron valores de igual forma excelentes de \textit{accuracy}, con todos los modelos obteniendo valores superior a 0.5, 4 de las 9 características teniendo valores sobre el 0.9 y un en promedio un valor de \textit{accuracy} de 0.80. Cabe resaltar que, en adición a estos resultados, este TFM es el primer trabajo que se conoce que ha utilizado ExMeshCNN para resolver un problema del mundo real haciendo uso de datos mucho más complejos y con mayor resolución de los que se utilizaron en la publicación original del \textit{framework}.

Asimismo, se empleó Grad-CAM para interpretar las regiones de la malla que influyeron en las predicciones de los modelos. Los resultados muestran que cada modelo se enfoca en zonas distintas del hueso, lo que sugiere que cada característica posee un patrón morfológico diferenciado. En la mayoría de los casos, las regiones de atención se concentraron en la cara anterior del pubis, lo cual coincide con la práctica común de los expertos, validando así que los criterios morfológicos de Todd corresponden efectivamente a patrones estructurales reales. Esto refuerza la idea de que la subjetividad del método de Todd puede mitigarse mediante su automatización. Otro punto interesante que añade peso a esta idea es que se encontraron similitudes entre las características que resultaron más sencillas y más complicadas de detectar por los modelos generados por ExMeshCNN y los forenses, naturalmente esta línea debe ser estudiada en mayor profundidad para obtener conclusiones avaladas por un análisis más detallado, pero resulta un indicio prometedor.

En conclusión, se han alcanzado satisfactoriamente los objetivos planteados, logrando entrenar un modelo por cada una de las características del método de Todd. Todo el código desarrollado se encuentra disponible públicamente en el repositorio de GitHub \url{https://github.com/RhinoBlindado/datcom-tfm}, a excepción del conjunto de datos empleado, cuya distribución está restringida por motivos de confidencialidad.

En definitiva, el trabajo desarrollado sienta una base sólida para futuras líneas de investigación multidisciplinares entre la AF y la IA. Las posibilidades de extensión son amplias: desde la incorporación de técnicas más sofisticadas de NAS, clasificación multietiqueta o regularización, hasta el aumento de la expresividad de los modelos mediante arquitecturas más profundas, mayor número de épocas o la integración de múltiples modelos especializados. La inclusión de información adicional, como las texturas óseas presentes en los escaneos 3D, también podría enriquecer notablemente el proceso de aprendizaje. Finalmente, una colaboración estrecha con expertos forenses permitiría no solo validar las interpretaciones generadas por técnicas como Grad-CAM, sino también identificar nuevas regiones morfológicas relevantes, reforzando así tanto la aplicabilidad práctica como el valor científico del enfoque propuesto.
\chapter{Conclusiones y Trabajos Futuros}

La estimación de la edad de la muerte constituye una parte esencial del estudio del perfil biológico y, al mismo tiempo, un problema complejo de gran relevancia para la antropología forense, debido a la subjetividad inherente a los métodos manuales actualmente empleados por los expertos. Este TFM abordó la clasificación automática de los criterios morfológicos utilizados para dicha estimación por medio de la sínfisis del pubis, con el objetivo de obtener modelos capaces de automatizar la extracción de estos criterios y asistir al experto humano en la toma de decisiones.

En primera instancia, se llevó a cabo un estudio detallado de la literatura sobre estimación de edad a partir de restos óseos, con foco en la sínfisis del pubis, así como del procesamiento de modelos 3D mediante \textit{Deep Learning} (Sección \ref{traditional_labeling_methods}). Se observó que la antropología forense continúa utilizando mayoritariamente el método de Todd \cite{RefWorks:RefID:19-todd1921age} o variantes del mismo, con escasa incorporación de innovaciones tecnológicas basadas en \textit{Deep Learning}. En cuanto al tratamiento de datos tridimensionales mediante técnicas de \textit{Deep Learning}, se constató la falta de consenso sobre su representación más adecuada, existiendo múltiples alternativas (Subsección \ref{3d_reprs_DL}). Para este trabajo se optó por el uso de mallas poligonales, por tratarse de un formato ampliamente utilizado en informática gráfica y que también se está introduciendo como una herramienta más robusta para el trabajo cotidiano de los antropólogos forenses. A su vez, no se hallaron estudios previos que aborden la clasificación directa y automática de las características morfológicas del pubis como los aquí planteados, por lo que este TFM se puede considerar original y novedoso.

Se exploraron diversas propuestas metodológicas basadas en mallas poligonales, y se seleccionó el enfoque ExMeshCNN \cite{kim_exmeshcnn_2022} como \textit{framework} base para el diseño de los modelos (Subsección \ref{section3:meshes}). Se aplicó la búsqueda de arquitectura neuronal mediante la librería Optuna para optimizar su estructura y demás hiperparámetros relevantes (Subsección \ref{section4:nas}).

El desarrollo de modelos implicó la creación de herramientas específicas para el preprocesamiento de los datos, principalmente la reducción de complejidad topológica mediante colapso de aristas, procurando minimizar la pérdida de información, así como el sellado automático de las mallas (Subsección \ref{data_preprocessing}). Como parte de los experimentos, se llevaron a cabo 83 ejecuciones que incluyeron un total de 10,800 entrenamientos de modelos entrenados en una etiqueta (Subsección \ref{unique_tag_exps}) y 6,000 entrenamientos con múltiples etiquetas por modelo (Subsección \ref{multi_tag_exps}), sumando 16,800 entrenamientos que necesitaron de aproximadamente 8,856 horas\footnote{Alrededor de 369 días de ejecución secuencial continua en una GPU Nvidia Tesla A100.} GPU\cite{englewood_understanding_2025}. Se obtuvieron así los 9 mejores modelos para cada una de las características del método de Todd, todos con puntuaciones de F1 Macro superiores a 0.6, alcanzando un F1 Macro promedio de 0.67 y superando el 0.7 en 4 de las 9 características, con una característica obteniendo un valor de 0.92. Cabe destacar que esta métrica (F1 Macro) es conservadora, y el análisis de las matrices de confusión evidencia que los modelos son capaces de clasificar correctamente la mayoría de las muestras, incluso considerando la limitada cantidad de datos y el altísimo desbalanceo entre clases. De igual modo, se obtuvieron valores excelentes de \textit{accuracy}, con todos los modelos obteniendo valores superior a 0.5, 4 de las 9 características teniendo valores sobre el 0.9 y en promedio un valor de \textit{accuracy} de 0.80. Cabe resaltar que, en adición a estos resultados, este TFM es el primer trabajo que se conoce que ha utilizado ExMeshCNN para resolver un problema del mundo real haciendo uso de datos mucho más complejos y con mayor resolución. Se utilizaron mallas de 25,000; 50,000 y 100,000 triángulos obtenidas por escaneos 3D, mientras que en la publicación original del \textit{framework} solo se utilizaron mallas de 500; 1,000 y 5,000 triángulos creadas a mano.

Asimismo, debido a que ExMeshCNN emplea convoluciones 1D, provee una versión de Grad-CAM para interpretar las regiones de la malla que influyeron en las predicciones de los modelos, coloreando en vez de píxeles, los triángulos con mayor respuesta en la malla (Subsección \ref{gradcam_analysis}). Los resultados muestran que cada modelo se enfoca en zonas distintas del hueso, lo que sugiere que cada característica posee un patrón morfológico diferenciado. En la mayoría de los casos, las regiones de atención se concentraron en la cara anterior del pubis, lo cual coincide con la práctica común de los expertos, validando así que los criterios morfológicos de Todd corresponden efectivamente a patrones estructurales reales. Esto refuerza la idea de que la subjetividad del método de Todd puede mitigarse mediante su automatización. Otro punto interesante que añade peso a esta idea es que se encontraron similitudes entre las características que resultaron más sencillas y más complicadas de detectar por los modelos generados por ExMeshCNN y los forenses. Naturalmente, esta línea debe ser estudiada en mayor profundidad para obtener conclusiones avaladas por un análisis más detallado, pero resulta un indicio prometedor.

En conclusión, se han alcanzado satisfactoriamente los objetivos planteados, logrando entrenar un modelo específico para cada una de las características morfológicas del método de Todd. Este resultado supone un avance significativo respecto al TFG previo, en el que se basa este TFM, en donde se desarrolló un único modelo empleando un \textit{framework} diferente (MeshCNN) y solo se entrenó para predecir una característica\footnote{El Nódulo Óseo (BN). En el TFG se obtuvo un \textit{accuracy} de 70\% y F1 de 0.70 mientras que en este TFM se obtiene un \textit{accuracy} de 91\% y F1 de 0.79 para dicha característica.}, utilizando un total de 32 entrenamientos con 98 mallas. En contraste, el presente TFM ha realizado una evaluación experimental mucho más intensa (16,800 entrenamientos distribuidos entre todas las características haciendo uso de 970 mallas), aplicando un \textit{framework} más potente y eficiente, empleando funciones de pérdida especializadas en datos con alto desbalanceo entre clases junto con técnicas de búsqueda de arquitectura neuronal, enfoques multietiqueta, y un preprocesado más sofisticado que incluye dos tipos de mallas y tres resoluciones diferentes. Todo el código desarrollado se encuentra disponible públicamente en el repositorio de GitHub \url{https://github.com/RhinoBlindado/datcom-tfm}, a excepción del conjunto de datos empleado, cuya distribución está restringida por motivos de confidencialidad.

El trabajo desarrollado obtuvo resultados tan prometedores que el siguiente paso es adaptar este TFM para ser sometido a una revista científica del área biomédica. Asimismo, sienta bases sólidas para futuras líneas de investigación multidisciplinares entre la antropología forense y la IA. Adicionalmente, las posibilidades de extensión son amplias: desde la incorporación de técnicas más sofisticadas de búsqueda de arquitectura neuronal, clasificación multietiqueta o regularización, hasta el aumento de la expresividad de los modelos mediante arquitecturas más profundas, mayor número de épocas o la integración de múltiples modelos especializados. La inclusión de información adicional, como las texturas óseas presentes en los escaneos 3D, también podría enriquecer notablemente el proceso de aprendizaje o se podría plantear la utilización de aprendizaje no supervisado para la obtención de las características. Finalmente, una colaboración estrecha con expertos forenses permitiría no solo validar las interpretaciones generadas por técnicas como Grad-CAM, sino también identificar nuevas regiones morfológicas relevantes, reforzando así tanto la aplicabilidad práctica como el valor científico del enfoque propuesto.
\chapter*{}
%\thispagestyle{empty}
%\cleardoublepage

%\thispagestyle{empty}

%\input{portada/portada_2}



%\cleardoublepage
\thispagestyle{empty}

\begin{center}
{\small \bfseries \myTitle}
\end{center}
\begin{center}
\myName
\end{center}

%\vspace{0.7cm}
\noindent{\textbf{Palabras clave}: aprendizaje profundo, visión por computador, antropología forense, estimación del perfil biológico, estimación de edad, clasificación, malla 3D.}

\noindent{\textbf{Resumen}}

La estimación de la edad de la muerte es una de las tareas más relevantes en antropología forense, al formar parte esencial del perfil biológico como paso previo fundamental para la identificación de personas fallecidas, especialmente cuando otros métodos como el ADN o las huellas dactilares no son viables. Uno de los métodos más utilizados, y del que han surgido un importante número de variantes, es el propuesto por Thomas Wingate Todd en 1921, basado en el análisis visual de la superficie de la sínfisis del pubis para identificar nueve características morfológicas asociadas al envejecimiento. Sin embargo, este proceso manual es altamente subjetivo y dependiente del juicio del experto.

Se aborda dicha problemática mediante la automatización del análisis morfológico de la sínfisis del pubis utilizando técnicas de aprendizaje profundo sobre mallas 3D de alta resolución. Se propone un enfoque novedoso usando el \textit{framework} de ExMeshCNN, arquitectura experimental de redes convolucionales diseñada para mallas 3D. El objetivo principal es la predicción automática de las nueve características definidas por el método de Todd. Siendo el primer trabajo en la aplicación innovadora de este \textit{framework} a una problemática real no existen arquitecturas preentrenadas para la misma. Por lo tanto, se recurre a la búsqueda automática de arquitecturas neuronales, dónde se han ejecutado más de 10,000 entrenamientos con distintas configuraciones. Además de la clasificación clásica, se explora la clasificación multietiqueta para aprovechar posibles dependencias entre características y mejorar el rendimiento predictivo.

Se dispone de un total de 970 escaneos 3D de la sínfisis del pubis. Datos que requirieron de limpieza, remuestreo y reparación iterativa para obtener mallas limpias y válidas para los modelos. Se evaluaron tres resoluciones de malla (25K, 50K y 100K triángulos), y dos configuraciones espaciales: la malla completa del hueso y una versión recortada centrada en la zona anatómicamente relevante.

Los resultados muestran que los modelos son capaces de aprender con éxito los patrones morfológicos relevantes, incluso con datos reales que poseen un altísimo desbalance de clases. Se observa una concordancia parcial entre los patrones detectados por los modelos y el comportamiento de los expertos humanos, reforzada además por las visualizaciones obtenidas mediante Grad-CAM, que muestran regiones diferenciadas para cada característica. Las dos mejores características alcanzan un F1 Macro de 0.92 y 0.86 respectivamente, con un valor medio de 0.67 y un 80.17\% de precisión media, con cuatro características superando el 90\% en la métrica.

Este trabajo demuestra la viabilidad de los modelos de aprendizaje profundo para procesar satisfactoriamente datos 3D para el análisis automatizado de estructuras óseas complejas, sentando las bases para futuras mejoras metodológicas y aplicaciones prácticas en el ámbito forense.

\newpage
\thispagestyle{empty}


\begin{center}
{\large\bfseries \myTitleENG} \\
\end{center}
\begin{center}
\myName \\
\end{center}

\vspace{0.7cm}
\noindent{\textbf{Keywords}: deep learning, computer vision, forensic anthropology, biological profile estimation, age estimation, classification, 3D mesh} \\

\vspace{0.7cm}
\noindent{\textbf{Abstract}} \\

Age-at-death estimation is one of the most relevant tasks in forensic anthropology, as it constitutes an essential component of the biological profile and a crucial preliminary step for the identification of deceased individuals when other methods such as DNA analysis or fingerprinting are not viable. One of the most widely used methods, from which a significant number of variants have emerged, is the one proposed by Thomas Wingate Todd in 1921. This method is based on the visual analysis of the surface of the pubic symphysis to identify nine morphological features associated with aging. However, this manual process is highly subjective and depends heavily on expert judgment.

This study addresses that issue by automating the morphological analysis of the pubic symphysis using deep learning techniques applied to high-resolution 3D meshes. A novel approach is proposed using the ExMeshCNN framework: an experimental convolutional neural network architecture designed specifically for 3D mesh data. The main objective is the automatic prediction of the nine features defined by Todd's method. As this is the first work to apply this innovative framework to a real-world problem, no pretrained architectures exist for this task. Therefore, neural architecture search was employed, with over 10,000 training runs executed under different configurations. In addition to classical single-label classification, multi-label classification was explored to exploit potential dependencies between features and improve predictive performance.

A total of 970 3D scans of the pubic symphysis were available. These data required cleaning, resampling and iterative repair in order to obtain clean and valid meshes for the models. Three mesh resolutions were evaluated (25K, 50K, and 100K triangles), along with two spatial configurations: the full bone mesh and a cropped version focused on the anatomically relevant area.

The results show that the models are capable of successfully learning relevant morphological patterns, even when dealing with real-world data characterized by severe class imbalance. A partial agreement is observed between the patterns detected by the models and the behavior of human experts, further supported by Grad-CAM visualizations that highlight distinct attention regions for each feature. The two best-performing features achieved Macro F1 scores of 0.92 and 0.86, respectively, with a mean value of 0.67 and an average accuracy of 80.17\%, with four features surpassing 90\%.

This work demonstrates the feasibility of using deep learning models to successfully process 3D data for the automated analysis of complex bone structures, laying the groundwork for future methodological improvements and practical applications in the forensic field.
\chapter*{}
\thispagestyle{empty}

\noindent\rule[-1ex]{\textwidth}{2pt}\\[4.5ex]

Yo, \textbf{\myName}, alumno de la titulación Máster de Ciencia de Datos e Ingeniería de Computadores de la \textbf{Escuela Técnica Superior
de Ingenierías Informática y de Telecomunicación de la Universidad de Granada}, con pasaporte \myDNI, autorizo la
ubicación de la siguiente copia de mi Trabajo Fin de Máster en la biblioteca del centro para que pueda ser
consultada por las personas que lo deseen.

\vspace{6cm}

\noindent Fdo: \myName

\vspace{2cm}

\begin{flushright}
Granada a X de X de 2025.
\end{flushright}


\chapter*{}
\thispagestyle{empty}

\noindent\rule[-1ex]{\textwidth}{2pt}\\[4.5ex]

D. \textbf{\myProf}, Profesor del Departamento de Lenguajes y Sistemas Informáticos de la Universidad de Granada.

\vspace{0.25cm}

D. \textbf{\myOtherProf}, Profesor del Departamento de Ciencias de la Computación e Inteligencia Artificial de la Universidad de Granada.


\vspace{0.25cm}

\textbf{Informan:}

\vspace{0.25cm}

Que el presente trabajo, titulado \textit{\textbf{\myTitle}},
ha sido realizado bajo su supervisión por \textbf{\myName}, y autorizamos la defensa de dicho trabajo ante el tribunal
que corresponda.

\vspace{0.5cm}

Y para que conste, expiden y firman el presente informe en Granada a X de X de 2025.

\vspace{0.5cm}

\textbf{Los directores:}

\vspace{5cm}

\noindent \textbf{\myProf \hfill
\myOtherProf}

\chapter*{Agradecimientos}
\thispagestyle{empty}

\vspace{1cm}

Quisiera agradecer profundamente a Sergio Damas y Pablo Mesejo, por nuevamente ser mis tutores para la realización de este Trabajo de Fin de Máster en esta línea de investigación que me resulta tan interesante. Resalto la grandísima paciencia que me habéis tenido, el ánimo que me habéis dado y todos los conocimientos que me habéis impartido en el transcurso de la realización de este proyecto.

De igual forma agradezco a mis padres, así como a mis queridos amigos por haberme apoyado, escuchado, alentado y sobre todo, acompañado en la realización de esta importantísima etapa formativa de mi vida.